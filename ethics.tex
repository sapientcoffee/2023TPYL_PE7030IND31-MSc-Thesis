\documentclass[jou, 12pt]{apa7}
\usepackage[british]{babel}
\usepackage{csquotes}

\usepackage{supertabular}
\usepackage{fancyhdr}
\usepackage{tabularx}
\usepackage{textgreek}
\usepackage{setspace}
\usepackage{pdfpages}
\usepackage{graphicx}
% \usepackage{fixltx2e}

\usepackage[style=apa,sortcites=true,sorting=nyt,backend=biber]{biblatex}


\begin{document}
\maketitle
\onecolumn

\section{Introduction}
% Approx 500 words
% Background and literature
% Why its important
% Justification for your approach
% Aims and hypotheses extended logically from background

The software engineering/development industry (Google DuetAI, Microsoft Copilot, OpenAI etc.) currently has a hype cycle around assisted development with generative artificial intelligence (GenAI) \parencite{Nguyen-Duc2023GenerativeAgenda}, in particular large language models (LLM’s) with claims of increased developer productivity, reduced cognitive effort and increased developer experience. The purpose of this MSc Thesis will investigate the impact on cognitive effort for software engineers when GenAI capabilities are embedded into their IDE (integrated development environment), therefore being an integral part of their workflow.

Developers/engineers apply cognitive effort to mentally process the structures of the source code (which is a cognitive process known as source code comprehension) \parencite{Crk2016AssessingComprehension, Crk2015UnderstandingExpertise}, , and on interpreting different abstraction levels of software artifacts \parencite{Minas2017NeurophysiologicalDevelopers}. The measurement of cognitive load generates valuable information (such as the level of expertise of developers) for software engineering purposes \parencite{Crk2016AssessingComprehension,], Crk2015UnderstandingExpertise}, e.g, for accounting developers programming experience, and classification of the perceived difficulty \parencite{Fritz2014UsingDevelopment, Fritz2016LeveragingProductivity} during a coding task.

A quantitative approach will be taken with recruits from the DevOps Research and Assessment (DORA) \parencite{Humble2018Accelerate:Organizations} community of practice \parencite{DORACommunity} and peer network.

The majority of research to measure cognitive load of software developers has leveraged objective physiological measurements (e.g. EEG [*] [*] , eye-tracking [*] or dilation, heart rate or fMRI) with controlled experiments and have not been survey based [*]. They have not looked at GenAI as a topic. Surveys provide an opportunity to collect the subjective measurements from a wide array of individuals, from very experienced engineers to those still learning including a mix of traditional software developers and those that do not identify traditionally as software developers (DevOPs, SRE, infrastructure admins). Currently no quality models exist to measure the developers’ cognitive load. Human-Computer Interaction research (HCI) research, the intersection of psychology, social science and computer science and technology(*), has been used to measure users' mental demands or cognitive workload in relation to interfaces between individuals and computers and traditionally leverages questionnaires [*]. The most used measure by HCI researchers and practitioners is questionnaire based [*] with NASA-TLX being the most popular.

\section{Method}
Detailed description of your research activities – Method section• Study Design – enough detail that someone can replicate your study
• Methods of data collection (e.g. survey, interview,experiment, observation, participatory).
• Procedure – how will you run your study?
• Analysis – how will you treat your data?


Sample Groups
• Who is completing your study?• Where are they located?
• How are you accessing your potential participants?• Any organisation affiliation – need supporting letters.
• The number of sample groups
     • e.g. patients with depression and healthy volunteers would be two sample groups;
• The size of each sample group;• Inclusion and exclusion criteria





Recruitment• Advertise the study• Will participants approach you?• Will you approach participants?• How will you engage participants in the study?




Measures that you will use• Describe the measures that you will use• These need to be described in detail• Reviewer may not know the specific measures you want to use• Do your measures align with your research question?• Are your measures appropriate for your participants?• Detailed description of what the participants will be asked to do i.e.data collection:• How many interviews/assessments,• When,• Where,• For how long.• Anything you generate yourself needs to be reviewed by ethics

Ethical Issues• Everyone needs to consider:• How will people know what you are asking them to do?• When are you asking them to consent?• Think about any points you haven’t covered already• Issues specific to your sample• Conflict of interests• Any sensitivity in the measures you are using• Appropriateness of how you ask about demographicquestions e.g. gender versus sex; socioeconomic group• Any considerations in your recruitment process



Researcher and Participant SafetyIssues• Cannot act as support for participants –researcher only.• Support services need to be appropriate• At least some support services need to be freelyavailable.• Support services need to be geographically accessible.• Mandatory reporting – may or may not be relevant



Anonymisation of data• Equally as relevant to qualitative and quantitativedata• Qualitative data can be difficult to de-identify• Why might that be?• Types of data:• Anonymous – cannot identify anyone• De-identified – you have removed emails and provided a number/code



BPS Human Research Ethics Code:• https://www.bps.org.uk/news-and-policy/bps-code-human-research-ethics• BPS Guidelines for internet mediated research:• https://www.bps.org.uk/news-and-policy/ethics-guidelines-internet-mediated-research


\end{document}